\documentclass{article}
\usepackage[utf8]{inputenc}
\usepackage{CJKutf8} % 用来支持中文
\usepackage{caption}
\usepackage{graphics}
\usepackage{array}
\usepackage{booktabs}
\usepackage{geometry}
\usepackage{amsmath} % 用来写数学公式
\geometry{left=3.2cm, right=3.2cm, top=2.0cm, bottom=2.0cm}
\usepackage{setspace}
\setstretch{1.2} 
\usepackage[colorlinks,linkcolor=blue]{hyperref}
\usepackage{amsthm}
\usepackage{amsfonts}
\usepackage{graphicx}
\usepackage{epstopdf}
\setcounter{secnumdepth}{0} % section不显示编号
\usepackage{color}
\definecolor{shadecolor}{rgb}{0.94, 0.94, 1.0}
\usepackage{framed}

\title{\textbf{Statistical Inference Assignment 4}}
\author{Junhao Yuan (20307130129)}
\date{\today}

\begin{document}
\begin{CJK}{UTF8}{gbsn}

    \maketitle
    \def \RR{{\mathbb R}}
    \def \EE{{\mathbb E}}
    \def \VV{{\mathbb V}}
    \def \II{{\mathbb I}}
    \def \NN{{\mathcal N}}


    % Problem 1
    \begin{shaded}
        \noindent\textsc{Problem 1.}\par
        Let $X_1,\ldots, X_n \mathop{\sim}\limits^{iid} \NN (\theta, a \theta^2)$, where $a$ is
        a known positive constant and $\theta > 0$.
        \begin{itemize}
            \item [(a)] Show that the parameter space does not contain a two-dimensional open set.
            \item [(b)] Show that the statistics $T = (\bar{X}, S^2)$ is a sufficient statistics for $\theta$,
                  but the family of distributions is not complete.
        \end{itemize}
    \end{shaded}
    \noindent\textsc{Solution.}\par
    \begin{itemize}
        \item [(a)] The parameter space $\Theta = \{(\theta, a\theta^2): \theta > 0, a >0  \} $ is a parabola in a two-dimensional space,
              which does not contains a two-dimensional open set.
        \item [(b)] The pdf of the sample is
              \begin{align}
                  f(\mathbf{X}, \theta) & = (2a\pi \theta^2) ^{-n/2} \textnormal{exp} \left (-\frac{nS^2 + n(\bar{x} - \theta)^2}{2a\theta^2}  \right ).\notag
              \end{align}
              Hence, by the factorization theorem, $T = (\bar{X}, S^2)$ is a sufficient statistics for $\theta$.

              To prove that the family is not complete, we just need to find a function $g(\bar{X}, S^2)$ satisfies
              $\EE(g(\bar{X}, S^2)) = 0$, but $g(\bar{X}, S^2)$ don't have to be zero identically.
              We have already known that $\frac{nS^2}{a\theta^2} \sim \chi_{n-1}^2$. If we define $Y = \frac{nS^2}{a\theta^2}$,
              then we have
              \begin{align}
                  \EE(\sqrt{Y}) & = \int_{0}^{\infty} y^{\frac{1}{2}} \, \frac{y^{\frac{n-1}{2}-1}e^{-\frac{y}{2}}}{2^{\frac{n-1}{2}}\Gamma(\frac{n-1}{2})}\, dx \notag \\
                                & =\frac{\sqrt{2} \Gamma(\frac{n}{2})}{\Gamma(\frac{n-1}{2})}.\notag
              \end{align}
              Hence,
              \begin{align}
                  \EE(S) & = \sqrt{\frac{a}{n}}\theta \EE(\sqrt{Y}) = \sqrt{\frac{2a}{n}}\frac{\Gamma(\frac{n}{2})\theta }{\Gamma(\frac{n-1}{2})}.\notag
              \end{align}
              Therefore, if we define
              \begin{align}
                  g(\bar{X}, S^2) & = \bar{X} - \sqrt{\frac{n}{2a}} \frac{\Gamma(\frac{n-1}{2})}{\Gamma(\frac{n}{2})}S ,\notag
              \end{align}
              then we have $\EE(g(\bar{X}, S^2)) = 0$, but $g(\bar{X}, S^2)$ doesn't have to be zero identically, which
              means this family of distribution is not complete.
    \end{itemize}



    % Problem 2
    \begin{shaded}
        \noindent\textsc{Problem 2.}\par
        Let $X_1,\ldots, X_n$ be a random sample from the following population:
        \begin{align}
            f(x,\theta) = \theta x^{\theta - 1}, \qquad 0 < x < 1, \qquad \theta > 0.
        \end{align}
        \begin{itemize}
            \item [(a)] Is $\sum_{i=1}^n X_i$ sufficient for $\theta$?
            \item [(b)] Find a complete sufficient statistics for $\theta$.
        \end{itemize}
    \end{shaded}
    \noindent\textsc{Solution.}\par
    \begin{itemize}
        \item [(a)]
              The pdf of the sample is
              \begin{align}
                  f(\mathbf{X}|\theta) & = \prod_{i=1}^n \theta x_i^{\theta - 1} \notag                                                                                                       \\
                                       & = \theta^n\cdot \textnormal{exp}\left (\theta \left (\sum_{i=1}^n \textnormal{ln}(x_i)\right )\right ) \left (\prod_{i=1}^n x_i\right )^{-1} ,\notag
              \end{align}
              which belongs to the exponential family and $n\geq 1$.
              Therefore,
              \begin{align}
                  T(\mathbf{X}) & = \sum_{i=1}^n \textnormal{ln}(X_i),\notag
              \end{align}
              is a sufficient statistics for $\theta$.
              Now we prove that $T(\mathbf{X})$ is also the minimal sufficient statistics.
              Since
              \begin{align}
                  \frac{f(\mathbf{X}|\theta)}{f(\mathbf{Y}|\theta)} & = \left ( \frac{\prod_{i=1}^n x_i}{\prod_{j=1}^n y_j} \right )^{\theta - 1}\notag \\
                                                                    & =e^{(\theta - 1)(T(\mathbf{X}) - T(\mathbf{Y}))}\notag
              \end{align}
              is a constant function of $\theta$ if and only if $T(\mathbf{X})=T(\mathbf{Y})$, $T(\mathbf{X})$ is the minimal sufficient statistics for $\theta$.

              However, $T(\mathbf{X}) = T(\mathbf{Y})$ does not implies that $\sum_{i=1} x_i = \sum_{j=1}^n y_j$, which means $\sum_{i=1}^n x_i$ is not sufficient.
        \item [(b)] Since the natural parameter space $\Theta = (0, \infty)$ contains an open set,
              $T(\mathbf{X})$ is also a complete statistics for $\theta$.
    \end{itemize}




    % Problem 3
    \begin{shaded}
        \noindent\textsc{Problem 3.}\par
        Suppose $X_1,\ldots, X_n$ are independently sampled from the following pmf
        \begin{align}
            P(X=k)= - \frac{1}{\textnormal{ln}(1-p)} \frac{p^k}{k}, \qquad 0 < p < 1, k=1, 2,\ldots
        \end{align}
        Use the method of moment to find an estimator for $p$.
    \end{shaded}
    \noindent\textsc{Solution.}\par
    Since
    \begin{align}
        \EE(X)   & = \sum_{i=1}^{\infty} -\frac{p^k}{\textnormal{ln}(1-p)} = -\frac{1}{\textnormal{ln}(1-p)} \frac{p}{1-p} , \notag    \\
        \EE(X^2) & = \sum_{i=1}^{\infty} -\frac{kp^k}{\textnormal{ln}(1-p)} = -\frac{1}{\textnormal{ln}(1-p)} \frac{p}{(1-p)^2},\notag
    \end{align}
    we have
    \begin{align}
        p = 1 - \frac{\EE(X)}{\EE(X^2)}.\notag
    \end{align}
    Thus, by the method of moment, we have
    \begin{align}
        \hat{p} & = 1 - \frac{m_1}{m_2},\notag
    \end{align}
    where
    \begin{align}
        m_1 = \frac{\sum_{i=1}^n X_i}{n}, \qquad m_2 = \frac{\sum_{i=1}^n X_i^2}{n}.\notag
    \end{align}


    % Problem 4
    \begin{shaded}
        \noindent\textsc{Problem 4.}\par
        Suppose $X_1,\ldots, X_n \mathop{\sim}\limits^{iid} \NN(\mu, \sigma^2)$. Find a method of moment
        estimator for $P(X>1)$.
    \end{shaded}
    \noindent\textsc{Solution.}\par
    Let $Y=\frac{X-\mu}{\sigma}$, then $P(X>1) = P(Y>\frac{1-\mu}{\sigma})= 1-\Phi\left (\frac{1-\mu}{\sigma}\right )$, where $\Phi(x)$ is the cdf of the standard normal distribution.
    By the method of moment, we have $\hat{\mu} = \bar{X}$ and $\hat{\sigma}^2 = S^2 = \sum_{i=1}^n (X_i - \bar{X})^2/n$.
    Therefore, a method of moment estimator of $P(X>1)$ is
    \begin{align}
        1 - \Phi\left (\frac{1-\bar{X}}{S}\right ).\notag
    \end{align}
\end{CJK}
\end{document}