\documentclass{article}
\usepackage[utf8]{inputenc}
\usepackage{CJKutf8} % 用来支持中文
\usepackage{caption}
\usepackage{graphics}
\usepackage{array}
\usepackage{booktabs}
\usepackage{geometry}
\usepackage{amsmath} % 用来写数学公式
\geometry{left=3.2cm, right=3.2cm, top=2.0cm, bottom=2.0cm}
\usepackage{setspace}
\setstretch{1.2} 
\usepackage[colorlinks,linkcolor=blue]{hyperref}
\usepackage{amsthm}
\usepackage{amsfonts}
\usepackage{graphicx}
\usepackage{epstopdf}
\setcounter{secnumdepth}{0} % section不显示编号
\usepackage{color}
\definecolor{shadecolor}{rgb}{0.94, 0.94, 1.0}
\usepackage{framed}

\title{\textbf{Statistical Inference Assignment 2}}
\author{Junhao Yuan (20307130129)}
\date{\today}

\begin{document}
\begin{CJK}{UTF8}{gbsn}

    \maketitle

    \newtheorem{lemma}{Lemma}[]

    % Problem 1
    \begin{shaded}
        \noindent\textsc{Problem 1.}\par
        Let $X_1, \ldots, X_n \mathop{\sim}\limits^{i.i.d.} N(\mu, \sigma^2)$, $\bar{X} = \frac{1}{n}\sum_{i=1}^nX_i$,
        $S_n^2=\frac{1}{n} \sum_{i=1}^n (X_i-\bar{X})^2$. Suppose $X_{n+1}\sim N(\mu, \sigma^2)$ and is independent of
        $X_1, \ldots, X_n$, find the distribution of
        \begin{align}
            \frac{X_{n+1}-\bar{X}}{S_n}\sqrt{\frac{n-1}{n+1}}
        \end{align}
    \end{shaded}
    \noindent\textsc{Solution.}\par
    Since $\bar{X}\sim N(\mu,\frac{\sigma^2}{n})$ and $X_{n+1}\sim N(\mu,\sigma^2)$, we have
    \begin{align}
        \frac{X_{n+1}-\bar{X}}{\sqrt{\frac{n+1}{n}}\sigma} \sim N(0,1),\notag
    \end{align}
    and
    \begin{align}
        \frac{nS_n^2}{\sigma^2}\sim \chi_{n-1}^2.\notag
    \end{align}
    Thus, by the definition of $t$-distribution, we can get that
    \begin{align}
        \frac{(X_{n+1}-\bar{X})/\sqrt{\frac{n+1}{n}\sigma^2}}{\sqrt{\frac{nS_n^2}{(n-1)\sigma^2}}}
        = \frac{X_{n+1}-\bar{X}}{S_n}\sqrt{\frac{n-1}{n+1}} \sim t_{n-1}.\notag
    \end{align}



    % Problem 2
    \begin{shaded}
        \noindent\textsc{Problem 2.}\par
        Suppose $X_1,\ldots, X_n$ are independent and $X_i\sim N(0,\sigma^2)$ for $i=1,\ldots,n$. Define
        \begin{align}
            \xi = \sum_{i=1}^n \frac{(X_i-Z)^2}{\sigma_i^2}
        \end{align}
        where
        \begin{align}
            Z = \left (\sum_{i=1}^n\frac{X_i}{\sigma_i^2} \right ) \left (\sum_{i=1}^n\frac{1}{\sigma_i^2} \right )^{-1}.
        \end{align}
        Find the distribution of $\xi$. (Hint: Use a proper orthogonal transform.)
    \end{shaded}
    \noindent\textsc{Solution.}\par
    Rewrite $\xi$ as
    \begin{align}
        \xi = \sum_{i=1}^n \left ( \frac{X_i}{\sigma_i}\right )^2 - \left (\sum_{i=1}^n \frac{1}{\sigma_i^2}\right ) Z^2. \label{xi}
    \end{align}
    We can find a matrix $A\in \mathbb{R}^{n\times n}$ in the form as
    \[
        \setlength{\arraycolsep}{5.5pt}
        \begin{bmatrix}
            \dfrac{1/\sigma_1}{\delta} & \dfrac{1/\sigma_2}{\delta} & \ldots & \dfrac{1/\sigma_n}{\delta} \\[7pt]
            a_{21}                     & a_{22}                     & \ldots & a_{2n}                     \\[7pt]
            \vdots                     & \vdots                     & \ddots & \vdots                     \\[7pt]
            a_{n1}                     & a_{n2}                     & \ldots & a_{nn}
        \end{bmatrix}
    \]
    with rank $n$, where $\delta = \sqrt{\sum_{i=1}^n \sigma_i^{-2}}$.
    Then we can use Schmidt orthogonalization to transform $A$ into an orthogonal matrix (keep the first row of $A$ unchanged).
    If we regard $X_i/\sigma_i$ as a new random variable $X^{\prime}$ and define $Y:=AX^{\prime}$,
    with some simple calculation, we have $Y_1=\delta Z$,
    and
    \begin{align}
        Y_i = \sum_{j=1}^n a_{ij} \frac{X_j}{\sigma_j},\, (i=2,\ldots,n).\notag
    \end{align}
    It's obvious that $\mathbb{E}(Y_i)=0$, and
    \begin{align}
        \mathbb{V}(Y_i) = \sum_{j=1}^n a_{ij}^2 = 1, \, (i=2.\ldots,n).\notag
    \end{align}
    Thus we have $Y_i \sim N(0,1)$ for $i=2,\ldots,n$.
    Plug the results we get into \eqref{xi} and we can get that
    \begin{align}
        \xi & = \sum_{i=1}^nY_i^2 - Y_1^2 \notag \\
            & = \sum_{i=2}^n Y_i^2, \notag
    \end{align}
    which means $\xi \sim \chi^2_{n-1}$.


    % Problem 3
    \begin{shaded}
        \noindent\textsc{Problem 3.}\par
        Suppose $X_1,\ldots,X_n\sim Poisson(\lambda)$. Show that
        \begin{align}
            \frac{\bar{X}-\lambda}{\sqrt{\bar{X}/n}} \mathop{\longrightarrow}\limits^DN(0,1)
        \end{align}
        (Hint: Use Slutsky's theorem.)
    \end{shaded}
    \noindent\textsc{Solution.}\par
    If we define $Y=\sum_{i=1}^n X_i$, then by the property of Poisson distribution,
    we know that $Y\sim Poisson(n\lambda)$, and $\mathbb{E}(Y)=\mathbb{V}(Y) = n\lambda$.
    Therefore, according to the CLT, we know that
    \begin{align}
        \frac{Y-n\lambda}{\sqrt{n\lambda}} \mathop{\longrightarrow}\limits^D N(0,1).\notag
    \end{align}
    Additionally, we have
    \begin{align}
        Y\mathop{\longrightarrow}\limits^P n\lambda,\notag
    \end{align}
    derived from LLN. And from Slutsky theorem, we know that
    \begin{align}
        \frac{\sqrt{n\lambda}}{\sqrt{Y}} \mathop{\longrightarrow}\limits^P 1,\notag
    \end{align}
    and
    \begin{align}
        \frac{Y-n\lambda}{\sqrt{n\lambda}} \cdot \frac{\sqrt{n\lambda}}{\sqrt{Y}} =\frac{\bar{X}-\lambda}{\sqrt{\bar{X}/n}}  \mathop{\longrightarrow}\limits^D N(0, 1).\notag
    \end{align}


    % Problem 4
    \begin{shaded}
        \noindent\textsc{Problem 4.}\par
        If $X$ is a random variable with pdf or pmf of the form
        \begin{align}
            f(x|\boldsymbol{\theta}) = h(x)c(\boldsymbol{\theta})exp\left ( \sum_{i=1}^k w_i(\boldsymbol{\theta})t_i(x)\right )
        \end{align}
        show that
        \begin{itemize}
            \item [(a)]:
                  \begin{align}
                      \mathbb{E}\left (\sum_{i=1}^k \frac{\partial w_i(\boldsymbol{\theta})}{\partial \theta_j}t_i(X) \right ) = -\frac{\partial}{\partial \theta_j}log\, c(\boldsymbol{\theta})\label{a}
                  \end{align}
            \item [(b)]:
                  \begin{align}
                      \mathbb{V}\left (\mathop{\sum}_{i=1}^k \frac{\partial w_i(\boldsymbol{\theta})}{\partial \theta_j }t_i(X) \right )
                      = -\frac{\partial^2}{\partial \theta_j^2}log\, c(\boldsymbol{\theta}) - \mathbb{E}\left (\sum_{i=1}^k \frac{\partial^2 w_i(\boldsymbol{\theta})}{\partial \theta_j^2} t_i(X) \right ).\label{b}
                  \end{align}
        \end{itemize}
        Hint:
        \begin{align}
            \int f(x|\boldsymbol{\theta}) dx = \int h(x)c(\boldsymbol{\theta})
            exp\left (\sum_{i=1}^k w_i(\boldsymbol{\theta})t_i(x) \right ) =1.\label{hint}
        \end{align}
        Differentiate both sides and then rearrange terms.
    \end{shaded}
    \noindent\textsc{Solution.}\par
    \begin{itemize}
        \item [(a)] Differentiate both sides of \eqref{hint} and we can get that
              \begin{align}
                  0 & = \int h(x) \left [\frac{\partial c(\boldsymbol{\theta})}{\partial \theta_j}
                      exp\left (\sum_{i=1}^kw_i(\boldsymbol{\theta})t_i(x) \right ) +
                      c(\boldsymbol{\theta})exp\left (\sum_{i=1}^kw_i(\boldsymbol{\theta})t_i(x) \right )
                      \sum_{i=1}^k\frac{\partial w_i(\boldsymbol{\theta})}{\partial \theta_j}t_i(x) \right ] dx.           \label{13}
              \end{align}
              Rearrange \eqref{13} and we have
              \begin{align}
                  0 & = \frac{1}{c(\boldsymbol{\theta})}\frac{\partial c(\boldsymbol{\theta})}{\partial \theta_j} +
                  \mathbb{E}\left (\sum_{i=1}^k\frac{\partial w_i(\boldsymbol{\theta})}{\partial \theta_j}t_i(X) \right )                                        \notag                                      \\
                    & = \frac{\partial }{\partial \theta_j}\, log(c(\boldsymbol{\theta})) +  \mathbb{E}\left (\sum_{i=1}^k\frac{\partial w_i(\boldsymbol{\theta})}{\partial \theta_j}t_i(X) \right ), \notag \\
              \end{align}
              which is equation \eqref{a}.
        \item [(b)] Differentiate both sides of \eqref{13}, and we have
              \begin{align}
                  0 & = \int h(x) \frac{\partial^2 c(\boldsymbol{\theta})}{\partial \theta_j^2}exp\left (\sum_{i=1}^kw_i(\boldsymbol{\theta})t_i(x) \right )dx                                                                            \notag              \\
                    & \qquad +2 \int h(x) \frac{\partial c(\boldsymbol{\theta})}{\partial \theta_j} exp\left (\sum_{i=1}^kw_i(\boldsymbol{\theta})t_i(x) \right )  \sum_{i=1}^k\frac{\partial w_i(\boldsymbol{\theta})}{\partial \theta_j}t_i(x)\, dx  \notag \\
                    & \qquad + \int h(x) c(\boldsymbol{\theta}) exp\left (\sum_{i=1}^kw_i(\boldsymbol{\theta})t_i(x) \right ) \left (\sum_{i=1}^k\frac{\partial w_i(\boldsymbol{\theta})}{\partial \theta_j}t_i(x)\right )^2\, dx                \notag       \\
                    & \qquad + \int h(x)c(\boldsymbol{\theta}) exp\left (\sum_{i=1}^kw_i(\boldsymbol{\theta})t_i(x) \right ) \sum_{i=1}^k \frac{\partial^2 w_i(\boldmath\theta)}{\partial \theta_j^2} t_i(x)\, dx                        \notag               \\
                    & =\frac{1}{c(\boldsymbol{\theta})} \frac{\partial^2 c(\boldsymbol{\theta})}{\partial \theta_j^2}
                  + \frac{2}{c(\boldsymbol{\theta})}\frac{\partial c(\boldsymbol{\theta})}{\partial \theta_j} \mathbb{E}\left (\sum_{i=1}^kw_i(\boldsymbol{\theta})t_i(X) \right )
                  + \mathbb{E}\left (\left (\sum_{i=1}^kw_i(\boldsymbol{\theta})t_i(X) \right )^2\right )                                                                                                                         \notag                      \\
                    & \qquad + \mathbb{E}\left (\sum_{i=1}^k \frac{\partial^2 w_i(\boldmath\theta)}{\partial \theta_j^2} t_i(X) \right )                                                                                         \notag                       \\
                    & = \frac{1}{c(\boldsymbol{\theta})} \frac{\partial^2 c(\boldsymbol{\theta})}{\partial \theta_j^2}
                  - \frac{1}{c^2(\boldsymbol{\theta})} \left (\frac{\partial c(\boldsymbol{\theta})}{\partial \theta_j} \right )^2
                  + \mathbb{V}\left (\sum_{i=1}^k \frac{\partial^2 w_i(\boldmath\theta)}{\partial \theta_j^2} t_i(X) \right )                                                                                                          \notag                 \\
                    & \qquad + \mathbb{E}\left (\sum_{i=1}^k \frac{\partial^2 w_i(\boldmath\theta)}{\partial \theta_j^2} t_i(X) \right )                                                                                         \notag                       \\
                    & = \frac{\partial^2}{\partial \theta_j^2}\, log(c(\boldsymbol{\theta}))
                  + \mathbb{V}\left (\sum_{i=1}^k \frac{\partial^2 w_i(\boldmath\theta)}{\partial \theta_j^2} t_i(X) \right )
                  + \mathbb{E}\left (\sum_{i=1}^k \frac{\partial^2 w_i(\boldmath\theta)}{\partial \theta_j^2} t_i(X) \right ).\notag
              \end{align}
              Thus, we can get \eqref{b}.
    \end{itemize}



\end{CJK}
\end{document}