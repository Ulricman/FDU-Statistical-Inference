\documentclass{article}
\usepackage[utf8]{inputenc}
\usepackage{CJKutf8} % 用来支持中文
\usepackage{caption}
\usepackage{graphics}
\usepackage{array}
\usepackage{booktabs}
\usepackage{geometry}
\usepackage{amsmath} % 用来写数学公式
\geometry{left=3.2cm, right=3.2cm, top=2.0cm, bottom=2.0cm}
\usepackage{setspace}
\setstretch{1.2} 
\usepackage[colorlinks,linkcolor=blue]{hyperref}
\usepackage{amsthm}
\usepackage{amsfonts}
\usepackage{lipsum}
\usepackage{framed}
\usepackage[strict]{changepage}
\usepackage{newtxtext}
\usepackage[dvipsnames,svgnames]{xcolor}
\usepackage{tcolorbox}
\usepackage{color}
\usepackage{cases}
\usepackage{indentfirst}


\title{\textbf{Statistical Inference}}
\author{}
\date{}

\begin{document}
\begin{CJK}{UTF8}{gbsn}

    \maketitle
    \newtheorem{Inference}{Theorem}[section]

    % Probability
    \section{Probability}

    % The derivation of the pdf of order statistics in a direct way
    \begin{Inference}
        (\textcolor{red}{\textit{The derivation of the pdf of order statistics in a direct way}})
        \begin{align}
            f_{x_{(r)}}(t)=\frac{n!}{(r-1)!(n-r)!}[F_x(t)]^{r-1}f_{x}(t)[1-F_x(t)]^{(n-r)}
        \end{align}
    \end{Inference}
    \begin{proof}
        By the definition of $F_{x(r)}(t)$, we have
        \begin{align}
            F_{x_{(r)}}(t) = \mathbb{P}(x_{(r)}\leq t) & = \sum_{i=r}^n\mathbb{P}(\sum_{j=1}^n I (x_j\leq t) = i)\notag         \\
                                                       & = \sum_{i=r}^n \binom{n}{i} [F_x(t)]^i[1-F_x(t)]^{n-i}.\label{order_1}
        \end{align}
        To get the pdf, we can take derivative of both sides of \eqref{order_1}
        \begin{align}
            f_{x_{(r)}}(t) & = \sum_{i=r}^n \binom{n}{i} \left \{ i[F_x(t)]^{i-1}f_x(t)[1-F_x(t)]^{n-i}-(n-i)[F_x(t)]^i[1-F_x(t)]^{n-i-1}f_x(t) \right \}\notag \\
                           & =\sum_{i=r}^n \binom{n}{i} [F_x(t)]^{i-1}[1-F_x(t)]^{n-i-1}f_x(t)\left \{ i[1-F_x(t)] - (n-i)F_x(t) \right \} \notag               \\
                           & = \binom{n}{r}r[F_x(t)]^{r-1}[1-F_x(t)]^{n-r} + \sum_{i=r+1}^n \binom{n}{i} i[F_x(t)]^{i-1}[1-F_x(t)]^{n-i} \notag                 \\
                           & \qquad - \sum_{i=r}^{n-1}\binom{n}{i}(n-i)[F_x(t)]^{i-1}[1-F_x(t)]^{n-i}\notag                                                     \\
                           & = \binom{n}{r}r[F_x(t)]^{r-1}[1-F_x(t)]^{n-r} + \sum_{i=r}^{n-1} \binom{n}{i+1} {(i+1)}[F_x(t)]^{i}[1-F_x(t)]^{n-i-1} \notag       \\
                           & \qquad - \sum_{i=r}^{n-1}\binom{n}{i}(n-i)[F_x(t)]^{i-1}[1-F_x(t)]^{n-i}\notag                                                     \\
                           & = f_{x_{(r)}}(t)=\frac{n!}{(r-1)!(n-r)!}[F_x(t)]^{r-1}f_{x}(t)[1-F_x(t)]^{(n-r)}\notag
        \end{align}
    \end{proof}

    \begin{Inference}
        (\textcolor{red}{\textit{the pdf of noncentral chi-squared distribution}})
        \begin{equation}
            f(x) =
            \begin{cases}
                e^{-\delta^2/2}\mathop{\sum}\limits_{i=0}^{\infty} \frac{1}{i!}\left ( \frac{\delta^2}{2} \right )^i \frac{x^{i+n/2-1}}{2^{i+n/2}\Gamma(n/2+i)}e^{-x/2}, & x > 0    \\
                0,                                                                                                                                                       & x \leq 0
            \end{cases}
            \label{noncentral}
        \end{equation}
        where $\delta = \sqrt{\mathop{\sum}\limits_{i=1}^n a_i^2}$.

    \end{Inference}
    \begin{proof}
        To prove \eqref{noncentral}, we need to use a orthogonal transformation such that
        $X_1^2+\cdots+X_n^2 = Y_1^2+Z$, where $X_i\sim N(a_i, 1)$, $Y_1\sim N(\delta, 1)$ and
        $Z\sim \chi_{n-1}^2$. The orthogonal matrix $A$ can be
        \[
            \setlength{\arraycolsep}{3pt}
            \begin{bmatrix}
                \frac{a_1}{\delta}                      & \frac{a_2}{\delta}                      & \cdots                                   & \frac{a_n}{\delta} \\
                \frac{a_2}{\sqrt{a_1^2+a_2^2}}          & \frac{-a_1}{a_1^2 + a_2^2}              &                                          &                    \\[7pt]
                \frac{a_2a_3}{\sqrt{\sum_{i=1}^3a_i^2}} & \frac{a_1a_3}{\sqrt{\sum_{i=1}^3a_i^2}} & \frac{-a_1a_2}{\sqrt{\sum_{i=1}^3a_i^2}} &                    \\[7pt]
                                                        &                                         &                                          & \ddots             \\
            \end{bmatrix}.
        \]
    \end{proof}

\end{CJK}
\end{document}
